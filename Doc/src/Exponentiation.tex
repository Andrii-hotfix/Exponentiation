%\usepackage{simplemargins}
%\usepackage[square]{natbib}
\documentclass{article}
\usepackage{amsmath}
\usepackage{amsfonts}
\usepackage{amssymb}
\usepackage{graphicx}
\usepackage{authblk}
\usepackage{hyperref}
\usepackage[utf8]{inputenc}
\usepackage[ukrainian]{babel}
\usepackage[backend=biber]{biblatex}
\addbibresource{article.bib}
\author{Дегтярьов Андрій, ФБ-91мп}
\begin{document}
\pagenumbering{gobble} %roman, arabic

\title{\textbf{МРМКЗІ: Лабораторна робота №1. Бібліотека багатослівної арифметики та алгоритми
швидкого піднесення в степінь }}

\date{15.05.2020}
\maketitle
\hspace{10pt}

\normalsize
\textbf{Вступ}  
У данній роботі було досліджено алгоритми піднесення в степінь чисел, які мають довільний розмір в бітах. 
Було імплементовано наступні алгоритми піднесення в степінь: 
\begin{list}
\item "SLiding Window "
\item "M-арний алгоритм"
\item "LR - бінарний алгоритм "
\item "RL - бінарний алгоритм "
\end{list}

Задля реалізації вище згаданих алгоритмів було реалізовано ряд допоміжних алгоритмів: арифметичні операції та
логічні операції.
Також було проведено тестування фунціоналу на основі реалізації багатослівної арифметики Gnu MP.
В решті, було проведено тестування швидкодії на данних різної розмірності для всіх 4х реалізованих 
алгоритмів та порівняно з бібліотечними реалізаціями GNU MP та OpenSSL.
Код тетів та безпосередньо бібліотеки був реалізований мовою С++.

\textbf{Binary LR}
Цей алгортм дуже простий 


\textbf{Тестування}
\begin{figure}[h!]
  \includegraphics[width=\linewidth]{"gr.png"}
  \caption{Порівняльні данні}
  \label{fig:cmp}
\end{figure}
\end{document}
